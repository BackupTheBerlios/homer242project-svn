Le travail en binome peut sembler assez d�routante au d�but. On peut penser que le travail � deux est une perte de temps et que si les 
deux programmeurs travaillaient chacun de son cot�, �a irait plus vite. L'avancement du projet se fait peut �tre moins rapidement mais les erreurs 
sont identifi�es plus rapidement. De plus, le fait d'�tre � deux permet d'�changer des id�es. Au final, c'est un "brainstorming" permanent.
Certaines �tudes ont �t� faite ainsi que des tests grandeur nature en universit� avec des �tudiants 
et il en r�sulte que les programmeurs sont plus efficaces lorsqu'ils communiquement et �changent leurs id�es.

- La programmation des tests est assez d�routante au d�but. On ne pense pas directement � coder des tests mais � �crire tout de suite 
notre id�e. Mais cette m�thode � plusieurs avantages :

    - Tout d'abord, elle permet de controler les fonctionnalit�s avant de programmer une fonctionnalit� qui les utilisent.
    - Elle permet aussi de voir comment le logiciel final va �tre programm�. Cet aspect est particuli�rement utile lorsqu'on programme 
    une API (ce qu'on a fait durant notre stage). On voit comment les objets vont interagir, si les commandes suivent une logique. On peut dej� 
    certains probl�mes � ce niveau l�.

- La phase d'analyse est courte. On propose rapidement une solution en construisant un diagramme UML simple (le plus souvent, un diagramme 
de classe suffit) sans y passer des semaines enti�res. Puis on teste l'impl�mentation. Cel� permet de tout de suite voir si la solution est bonne.
