% le * permet de ne pas num�roter le titre ni de le mettre dans la table des mati�res
\chapter*{Introduction}
\addcontentsline{toc}{chapter}{Introduction}
Le stage en entreprise est un des modules obligatoires pour la validation de la 3\ieme ann�e de l'\textsc{iup} g�nie math�matiques et informatique.
Ce module s'inscrit, parmi d'autre, dans la politique g�n�rale des l'\textsc{iup} qui est de fournir des maitres-ing�nieurs qualifi�s, autonomes et ayant une bonne connaissance des besoins des entreprises.

\section*{But du stage}
\addcontentsline{toc}{section}{But du stage}
Pour les �tudiants, le stage s'av�re une bonne batterie de test de certaines notions acquises que se soit � l'\textsc{iup} ou lors d'autre cursus. Le stage permet aussi de se familiariser avec le monde du travail. Il ne faut pas oublier qu'il est source d'apprentissage puisque il cr�e un climat de communication et d'�change entre plusieurs personnes (encadrants, stagiaires et autres interveants).

\section*{O�~?}
\addcontentsline{toc}{section}{O�~?}
Le stage s'est d�roul� au sein du service technique du d�partement informatique de l'\textsc{ufr} des sciences de l'universit� de 
Montpellier II. L'activit� de l'�quipe technique va du d�veloppement d'applications pour le d�partement � la gestion et la maintenance du parc informatique. Ce parc est constitu� de 17 salles situ�es au b�timent 6 de l'universit�. 
Au total ce sont plus de 300 machines accessibles aux �tudiants de l'\textsc{ufr}.

\section*{Sujet du stage}
\addcontentsline{toc}{section}{Sujet du stage}
le sujet du stage est la mise au point et la r�alisation de briques qui permettront d'installer un syst�me d'information 
tr�s rapidement et de le maintenir facilement. Il s'inscrit dans le cadre du d�veloppement d'une plate-forme web pr�sentant l'offre d'enseignement de l'\textsc{ufr} des sciences.

\section*{Chronologie des �v�nements}
\addcontentsline{toc}{section}{Chronologie des �v�nements}
- prise en main et re�criture d'adbapi  -----> petite participation dans le projet PIE ------> soumission des modifications ------> prise en compte des suggestion dans adbapi v0.2 -----> restructuration d'adbapi  ------> sortie d'adbapi v0.3 ( 1 mois )
- r�flexion sur les briques de bases -------> mise en place du m�canisme de r�le -----> retour � adbapi ---------> retour aux r�les

