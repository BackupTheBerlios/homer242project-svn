le cadre du stage
Le stage en entreprise est un des modules obligatoires pour la validation de la 3�me ann�e de l'I.U.P. G�nie Math�matiques et Informatique. il s'inscrit, parmi d'autre, dans la politique g�n�rale des l'I.U.P. qui est de fournir des maitres-ing�nieurs qualifi�s, autonomes et ayant une bonne connaissance des besoins des entreprises.

pourquoi ? (le but du stage)
Pour les �tudiants de l'I.U.P. le stage s'av�re une bonne batterie de test de certaines notions acquises que se soit � l'I.U.P. ou lors d'autre cursus. Le stage permet aussi de se familiariser avec le monde du travail. Il ne faut pas oublier qu'il est source d'apprentissage puisque il cr�e un climat de communication entre plusieurs personnes (encadrants stagiaires et autres individus).

o� ?
Au sein du service informatique de l'UFR des sciences de l'Universit� de Montpellier II.

quand ?
Du d�but du mois de Septembre jusqu'� la fin du mois de d�cembre.

le sujet du stage
le stage a pour sujet : la r�alisation et la mise au point de briques �l�mentaires g�n�riques n�cessaires � la mise en place d'application web. Il s'inscrit dans le cadre du d�veloppement d'une plate-forme Web pr�sentant l'offre d'enseignement de l'UFR des sciences.

chronologie des �v�nements
- prise en main et r��criture d'adbapi  -----> petite participation dans le projet PIE ------> soumission des modifications ------> prise en compte des suggestion dans adbapi v0.2 -----> restructuration d'adbapi en un "vrai" programme objet ------> validation ------> sortie d'adbapi v0.3 ( 1 mois )
- r�flexion sur les briques de bases -------> mise en place du m�canisme de r�le -----> retour � adbapi ........
